% arara: lualatex: { interaction: nonstopmode, synctex: no }
% arara: makeglossaries
% arara: lualatex: { interaction: nonstopmode, synctex: no }
% arara: lualatex: { interaction: nonstopmode, synctex: no }
\documentclass[a4paper,12pt,chapterprefix=false,bibliography=totoc,listof=totoc,]{scrreprt}

\usepackage{latex-style}

%%%%%%%%%%%%%%%%%%%%%%%%%%%%%%%%%%%%%%%%%%%%%%%%%%%%%%%%%%%%%%%%%%%%%%%%%%%%%%%
%% Linkeinstellungen
%%%%%%%%%%%%%%%%%%%%%%%%%%%%%%%%%%%%%%%%%%%%%%%%%%%%%%%%%%%%%%%%%%%%%%%%%%%%%%%

\hypersetup{
    colorlinks=true,
    linkcolor=blue,
    filecolor=magenta,      
    urlcolor=cyan,
}

\setlength{\parindent}{0pt}

%%%%%%%%%%%%%%%%%%%%%%%%%%%%%%%%%%%%%%%%%%%%%%%%%%%%%%%%%%%%%%%%%%%%%%%%%%%%%%%

% arara: halt
\newabbreviation{k8s}{K8S}{Kubernetes}
\newabbreviation{aws}{AWS}{Amazon Web Services}
\newabbreviation{tbd}{TBD}{To be determined}
\newabbreviation{na}{N/A}{Not applicable}
\newabbreviation{dto}{DTO}{Data Transferable Object}
\newabbreviation{stc}{STC}{Subject to change}
\newabbreviation{gcs}{GCS}{Google Cloud Services}
\newabbreviation{ucd}{UCD}{Use Case Diagramm}
\newabbreviation{mvc}{MVC}{Model-View-Controller pattern}
\newabbreviation{mvvm}{MVVM}{Model-View-Viewmodel pattern}

\makeglossaries

\begin{document}	
\begin{flushright}
GameBase
\\
Software Architecture Documentation
% \\
% For <Subsystem or Feature>
\bigbreak
Version 1.0
\end{flushright}
\chapter*{Revision History}
\begin{table}[H]
	\centering
	\everyrow{\hline}
	\begin{tabu} to \textwidth {|X[c]|X[c]|X[c]|X[c]|}
		Date & Version & Description & Author\\
		12/02/2019 & 0.1 & Create initial template & Stefan Lukas \\
		12/02/2019 & 0.5 & Create initial architecture documentation & Norman Gehrsitz \\
		% <mm/dd/yyyy> & <x.x> & <details> & <name>\\
	\end{tabu}
	\label{tab:rev-hist}
\end{table}

\tableofcontents

\chapter{Introduction}
%[The introduction of the Software Architecture Document provides an overview of the entire Software Architecture Document. It includes the purpose, scope, definitions, acronyms, abbreviations, references, and overview of the Software Architecture Document.]

\section{Purpose}
% [This section defines the role or purpose of the Software Architecture Document, in the overall project documentation, and briefly describes the structure of the document. The specific audiences for the document is identified, with an indication of how they are expected to use the document.]
This document provides a comprehensive architectural overview of the system, using a number of different architectural views to depict different aspects of the system. It is intended to capture and convey the significant architectural decisions which have been made on the system.

\section{Scope}
% [A brief description of what the Software Architecture Document applies to; what is affected or influenced by this document.]
This document describes the technical architecture of the GameBase project, including the Go Backend as well as the Angular Frontend.

\section{Definitions, Acronyms and Abbreviations}
\printabbreviations[title={}]

\section{References}
% [This subsection provides a complete list of all documents referenced elsewhere in the Software Architecture Document. Identify each document by title, report number (if applicable), date, and publishing organization. Specify the sources from which the references can be obtained. This information may be provided by reference to an appendix or to another document.]
\begin{table}[H]
	\centering
	\everyrow{\hline}
	\begin{tabu} to \textwidth {|X[c]|X[c]|X[c]|}
		\textbf{Title} & \textbf{Date} & \textbf{Author} \\
		\href{https://gitlab.tandashi.de/GameBase}{Git Repository} & 10/23/2019 & GameBase \\
		\href{https://youtrack.gahr.dev}{YouTrack} & 10/23/2019 & GameBase \\
		\href{https://gahr.dev}{Blog} & 10/23/2019 & GameBase \\
		\href{https://www.docker.com/}{Docker} & 10/23/2019 & Docker Inc. \\
		\href{https://cloud.google.com/}{Google Cloud Services} & 10/31/2019 & Google Ltd. \\
	\end{tabu}
	\label{tab:references-tabview}
\end{table}

\section{Overview}
% [This subsection describes what the rest of the Software Architecture Document contains and explains how the Software Architecture Document is organized.]
The next chapters provide information from different perspectives about the system architecture chosen to fulfill the software requirements.

\chapter{Architectural Representation}
% [This section describes what software architecture is for the current system, and how it is represented. Of the Use-Case, Logical, Process, Deployment, and Implementation Views, it enumerates the views that are necessary, and for each view, explains what types of model elements it contains.]
Our architecture uses the classical MVC Pattern for the Angluar Frontend as well as the Go backend. To reduce the implementation effort of a sharing the Model(representation of the Kubernetes Data) we use swagger.io.
This is a visual example of our overall Model:
%PNG


\chapter{Architectural Goals and Constraints}
% [This section describes the software requirements and objectives that have some significant impact on the architecture; for example, safety, security, privacy, use of an off-the-shelf product, portability, distribution, and reuse. It also captures the special constraints that may apply: design and implementation strategy, development tools, team structure, schedule, legacy code, and so on.]
By using a MVC architecture for both the Front- and Backend we gain various benefits, like productivity improvements by auto-generating the Code for our Model implementation in Go and Angular.
\subsection{Frontend}
Since our Frontend is writtien in Angular it internally also features a MVC Architecture.
\begin{itemize}
	\item Model: kubernetes data classes auto-gernerated by swagger
	\item View: HTML+CSS in the Browser
	\item Controller: Generated Javascript
\end{itemize}
\subsection{Backend}
Our Go Backend also leverages the advantages of a MVC design.
\begin{itemize}
	\item Model: kubernetes data classes auto-gernerated by swagger
	\item View: not applicable since we are only offfering a REST API
	\item Controller: gin HTTP Request Controller and kubernetes access logic
\end{itemize}
\chapter{Use-Case View}
% [This section lists use cases or scenarios from the use-case model if they represent some significant, central functionality of the final system, or if they have a large architectural coverage—they exercise many architectural elements or if they stress or illustrate a specific, delicate point of the architecture.]
This is the \gls{ucd} of our project:
\begin{figure}
	\includegraphics[width=\textwidth]{diagramms/Use_Case_Diagramm.png}
	\label{fig:ucd}
\end{figure}

%\section{Use-Case Realizations}
% [This section illustrates how the software actually works by giving a few selected use-case (or scenario) realizations, and explains how the various design model elements contribute to their functionality.]
%\gls{na}

\chapter{Logical View}
% [This section describes the architecturally significant parts of the design model, such as its decomposition into subsystems and packages. And for each significant package, its decomposition into classes and class utilities. You should introduce architecturally significant classes and describe their responsibilities, as well as a few very important relationships, operations, and attributes.]

\section{Overview}
% [This subsection describes the overall decomposition of the design model in terms of its package hierarchy and layers.]

\section{Architecturally Significant Design Packages}
% [For each significant package, include a subsection with its name, its brief description, and a diagram with all significant classes and packages contained within the package.

\chapter{Process View}
% [This section describes the system's decomposition into lightweight processes (single threads of control) and heavyweight processes (groupings of lightweight processes). Organize the section by groups of processes that communicate or interact. Describe the main modes of communication between processes, such as message passing, interrupts, and rendezvous.]
\gls{na}

\chapter{Deployment View}
% [This section describes one or more physical network (hardware) configurations on which the software is deployed and run. It is a view of the Deployment Model. At a minimum for each configuration it should indicate the physical nodes (computers, CPUs) that execute the software and their interconnections (bus, LAN, point-to-point, and so on.) Also include a mapping of the processes of the Process View onto the physical nodes.]
This is our deployment view including the Kubernetes cluster which is being accessed from our backend.
\begin{figure}
	\includegraphics[width=\textwidth]{diagramms/DeploymentView.png}
	\label{fig:ucd}
\end{figure}

\chapter{Implementation View}
% [This section describes the overall structure of the implementation model, the decomposition of the software into layers and subsystems in the implementation model, and any architecturally significant components.]
\gls{na}

%\section{Overview}
% [This subsection names and defines the various layers and their contents, the rules that govern the inclusion to a given layer, and the boundaries between layers. Include a component diagram that shows the relations between layers. ]
%\gls{na}

%\section{Layers}
% [For each layer, include a subsection with its name, an enumeration of the subsystems located in the layer, and a component diagram.]
%\gls{na}

%\chapter{Data View (optional)}
% [A description of the persistent data storage perspective of the system. This section is optional if there is little or no persistent data, or the translation between the Design Model and the Data Model is trivial.]

\chapter{Size and Performance}
% [A description of the major dimensioning characteristics of the software that impact the architecture, as well as the target performance constraints.]
Our Software acts as a Middleman between the Users Frontend and a Kubernetes Cluster. As such performance mainly depends on the Kubernetes Cluster where Containers are deployed to. Furthermore Go is a well optimized and multitasking oriented language. So even having multiple users send configuration requests to our backend should be no concern regarding performance.

\chapter{Quality}
% [A description of how the software architecture contributes to all capabilities (other than functionality) of the system: extensibility, reliability, portability, and so on. If these characteristics have special significance, such as safety, security or privacy implications, they must be clearly delineated.]

\gls{tbd}

\chapter{Supporting Information}
% [The supporting information makes the SRS easier to use.  It includes:
% \begin{itemize}
% 	\item Table of contents
% 	\item  Index
% 	\item Appendices
% \end{itemize}
% These may include use-case storyboards or user-interface prototypes. When appendices are included, the SRS should explicitly state whether or not the appendices are to be considered part of the requirements.]
If you want to reach out to use, feel free to write us an e-mail:
\begin{itemize}
	\item \href{mailto:gahr.leonhard@student.dhbw-karlsruhe.de}{Leonhard Gahr}
	\item \href{mailto:gehrsitz.norman@student.dhbw-karlsruhe.de}{Norman Gehrsitz}
	\item \href{mailto:lukas.stefan@student.dhbw-karlsruhe.de}{Stefan Lukas}
	\item \href{mailto:reis.kevin@student.dhbw-karlsruhe.de}{Kevin Reis}
\end{itemize}

Otherwise, take a look at section \ref{tab:references-tabview} for other possibilities to reach us out.
\end{document}