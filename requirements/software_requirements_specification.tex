\documentclass[a4paper,12pt,chapterprefix=false,bibliography=totoc,listof=totoc]{scrreprt}

\usepackage{vhistory}

\begin{document}
\begin{flushright}
<Project Name>
\\
Software Requirements Specification
\\
For <Subsystem or Feature>
\bigbreak
Version <1.0>
\end{flushright}
\begin{versionhistory}
	\vhEntry{1.0}{22.01.04}{JPW|KW}{created}
	\vhEntry{1.1}{23.01.04}{DP|JPW}{correction}
	\vhEntry{1.2}{03.02.04}{DP|JPW}{revised after review}
\end{versionhistory}

\chapter{Introduction}
\begin{verbatim}
[The introduction of the Software Requirements Specification (SRS) should provide an overview of the entire SRS. It should include the purpose, scope, definitions, acronyms, abbreviations, references, and overview of the SRS.]

[Note: The Software Requirements Specification (SRS) captures the complete software requirements for the system, or a portion of the system.  Following is a typical SRS outline for a project using only traditional natural-language style requirements – with no use-case modeling.  It captures all requirements in a single document,  with  applicable sections inserted from the  Supplementary Specifications (which would no longer be needed).  For a template of an SRS using use-case modeling, which consists of a package containing Use-Cases of the use-case model and applicable Supplementary Specifications and other supporting information, see rup_SRS-uc.dot.]

[Many different arrangements of an SRS are possible.  Refer to [IEEE830-1998] for further elaboration of these explanations, as well as other options for SRS organization.]
\end{verbatim}

\section{Purpose}
\begin{verbatim}
[Specify the purpose of this SRS. The SRS should fully describe the external behavior of the application or subsystem identified. It also describes nonfunctional requirements, design constraints and other factors necessary to provide a complete and comprehensive description of the requirements for the software.]
\end{verbatim}

\section{Scope}
\begin{verbatim}
[A brief description of the software application that the SRS applies to; the feature or other subsystem grouping; what Use-Case model(s) it is associated with;  and anything else that is affected or influenced by this document.]
\end{verbatim}

\section{Definitions, Acronyms and Abbreviations}
\begin{verbatim}
[This subsection should provide the definitions of all terms, acronyms, and abbreviations required to properly interpret the SRS.  This information may be provided by reference to the project Glossary.]
\end{verbatim}

\section{References}
\begin{verbatim}
[This subsection should provide a complete list of all documents referenced elsewhere in the SRS.  Each document should be identified by title, report number (if applicable), date, and publishing organization.  Specify the sources from which the references can be obtained. This information may be provided by reference to an appendix or to another document.]
\end{verbatim}

\section{Overview}
\begin{verbatim}
[This subsection should describe what the rest of the SRS contains and explain how the document is organized.]
\end{verbatim}

\chapter{Overall Description}
\begin{verbatim}
[This section of the SRS should describe the general factors that affect the product and its requirements.  This section does not state specific requirements.  Instead, it provides a background for those requirements, which are defined in detail in Section 3, and makes them easier to understand. Include such items as:
\end{verbatim}
\begin{itemize}
	\item product perspective
	\item product functions
	\item user characteristics
	\item constraints
	\item assumptions and dependencies
	\item requirements subsets
\end{itemize}

\chapter{Specific Requirements}
\begin{verbatim}
[This section of the SRS should contain all the software requirements to a level of detail sufficient to enable designers to design a system to satisfy those requirements, and testers to test that the system satisfies those requirements.   When using use-case modeling, these requirements are captured in the Use-Cases and the applicable supplementary specifications.  If use-case modeling is not used, the outline for supplementary specifications may be inserted directly into this section, as shown below.]
\end{verbatim}

\section{Functionality}
\begin{verbatim}
[This section describes the functional requirements of the system for those requirements which are expressed in the natural language style. For many applications, this may constitute the bulk of the SRS Package and thought should be given to the organization of this section. This section is typically organized by feature, but alternative organization methods may also be appropriate, for example, organization by user or organization by subsystem.  Functional requirements may include feature sets, capabilities, and security.

Where application development tools, such as requirements tools, modeling tools, etc., are employed to capture the functionality, this section document will refer to the availability of that data, indicating the location and name of the tool that is used to capture the data.]
\end{verbatim}

\subsection{<Functional Requirement One>}
\begin{verbatim}
[The requirement description.]
\end{verbatim}

\section{Usability}
\begin{verbatim}
[This section should include all of those requirements that affect usability. For example,
\end{verbatim}
\begin{itemize}
	\item specify the required training time for a normal users and a power user to become productive at particular operations
	\item specify measurable task times for typical tasks or base the new system’s usability requirements on other systems that the users know and like
	\item specify requirement to conform to common usability standards, such as IBM’s CUA standards Microsoft’s GUI standards]
\end{itemize}

\subsection{<Usability Requirement One>}
\begin{verbatim}
[The requirement description goes here.]
\end{verbatim}

\section{Reliability}
\begin{verbatim}
[Requirements for reliability of the system should be specified here. Some suggestions follow:
\end{verbatim}
\begin{itemize}
	\item Availability—specify the percentage of time available ( xx.xx%), hours of use, maintenance access, degraded mode operations, etc.
	\item Mean Time Between Failures (MTBF) — this is usually specified in hours, but it could also be specified in terms of days, months or years.
	\item Mean Time To Repair (MTTR)—how long is the system allowed to be out of operation after it has failed?
	\item Accuracy—specify precision (resolution) and accuracy (by some known standard) that is required in the system’s output.
	\item Maximum Bugs or Defect Rate—usually expressed in terms of bugs per thousand of lines of code (bugs/KLOC) or bugs per function-point( bugs/function-point).
	\item Bugs or Defect Rate—categorized in terms of minor, significant, and critical bugs: the requirement(s) must define what is meant by a “critical” bug; for example, complete loss of data or a complete inability to use certain parts of the system’s functionality.]
\end{itemize}

\subsection{<Reliability Requirement One>}
\begin{verbatim}
[The requirement description.]
\end{verbatim}

\section{Performance}
\begin{verbatim}
[The system’s performance characteristics should be outlined in this section. Include specific response times. Where applicable, reference related Use Cases by name.
\end{verbatim}
\begin{itemize}
	\item response time for a transaction (average, maximum)
	\item throughput, for example, transactions per second
	\item capacity, for example, the number of customers or transactions the system can accommodate
	\item degradation modes (what is the acceptable mode of operation when the system has been degraded in some manner)
	\item resource utilization, such as memory, disk, communications, etc.
\end{itemize}

\subsection{<Performance Requirement One>}
\begin{verbatim}
[The requirement description goes here.]
\end{verbatim}

\section{Supportability}
\begin{verbatim}
[This section indicates any requirements that will enhance the supportability or maintainability of the system being built, including coding standards, naming conventions, class libraries, maintenance access, maintenance utilities.]
\end{verbatim}

\subsection{<Supportability Requirement One>}
\begin{verbatim}
[The requirement description goes here.]
\end{verbatim}

\section{Design Constraints}
\begin{verbatim}
[This section should indicate any design constraints on the system being built. Design constraints represent design decisions that have been mandated and must be adhered to.  Examples include software languages, software process requirements, prescribed use of developmental tools, architectural and design constraints, purchased components, class libraries, etc.]
\end{verbatim}

\subsection{<Design Constraint One>}
\begin{verbatim}
[The requirement description goes here.]
\end{verbatim}

\section{On-line User Documentation and Help System Requirements}
\begin{verbatim}
[Describes the requirements, if any, for on-line user documentation, help systems, help about notices, etc.]
\end{verbatim}

\section{Purchased Components}
\begin{verbatim}
[This section describes any purchased components to be used with the system, any applicable licensing or usage restrictions, and any associated compatibility and interoperability or interface standards.]
\end{verbatim}

\section{Interfaces}
\begin{verbatim}
[This section defines the interfaces that must be supported by the application. It should contain adequate specificity, protocols, ports and logical addresses, etc. so that the software can be developed and verified against the interface requirements.]
\end{verbatim}

\subsection{User Interfaces}
\begin{verbatim}
[Describe the user interfaces that are to be implemented by the software.]
\end{verbatim}

\subsection{Hardware Interfaces}
\begin{verbatim}
[This section defines any hardware interfaces that are to be supported by the software, including logical structure, physical addresses, expected behavior, etc. ]
\end{verbatim}

\subsection{Software Interfaces}
\begin{verbatim}
[This section describes software interfaces to other components of the software system. These may be purchased components, components reused from another application or components being developed for subsystems outside of the scope of this SRS but with which this software application must interact.]
\end{verbatim}

\subsection{Communications Interfaces}
\begin{verbatim}
[Describe any communications interfaces to other systems or devices such as local area networks, remote serial devices, etc.]
\end{verbatim}

\section{Licensing Requirements}
\begin{verbatim}
[Defines any licensing enforcement requirements or other usage restriction requirements that are to be exhibited by the software.]
\end{verbatim}

\section{Legal, Copyright, and Other Notices}
\begin{verbatim}
[This section describes any necessary legal disclaimers, warranties, copyright notices, patent notice, wordmark, trademark, or logo compliance issues for the software.]
\end{verbatim}

\section{Applicable Standards}
\begin{verbatim}
[This section describes by reference any applicable standard and the specific sections of any such standards which apply to the system being described. For example, this could include legal, quality and regulatory standards, industry standards for usability, interoperability, internationalization, operating system compliance, etc.]
\end{verbatim}

\chapter{Supporting Information}
\begin{verbatim}
[The supporting information makes the SRS easier to use.  It includes:
\end{verbatim}
\begin{itemize}
	\item Table of contents
	\item  Index
	\item Appendices
\end{itemize}
\begin{verbatim}
These may include use-case storyboards or user-interface prototypes. When appendices are included, the SRS should explicitly state whether or not the appendices are to be considered part of the requirements.]
\end{verbatim}
\end{document}