\documentclass[a4paper,12pt,chapterprefix=false,bibliography=totoc,listof=totoc,book]{scrreprt}

% Benannte Farben
\usepackage{xcolor}
% Schriftauswahl
\usepackage{fontspec}
% Spracheinstellungen
\usepackage[english]{babel}
% Grafiken einfügen
\usepackage{graphicx}
% [H] Platzierung
\usepackage{float}
% Tabellen
\usepackage{tabu}
% KOMA-Script Mods für float,hyperref,listings,setspace
\RequirePackage{scrhack}

\setlength{\parindent}{0pt}

\begin{document}
\begin{flushright}
GameBase
\\
Use-Case Specification: Create GameServer
% \\
% For <Subsystem or Feature>
\bigbreak
Version 1.0
\end{flushright}

\tableofcontents

\chapter{Use-Case: Create GameServer}

\section{Brief Description}
The use case describes the creation, reading, updating and deleting of gameservers (CURD)

\chapter{Flow of Events}
[Add Activity Diagram]
\section{Basic flow}

\begin{itemize}
    \item User clicks on <<Create GameServer>>
    \item User fills the form with the required options and parameteres
    \item User clicks on <<Create>> to create the GameServer and gets redirected to the gameservers page
    \item User clicks on <<Cancel>> to cancel the Create operation
\end{itemize}

\section{Creation}
[include image]
The creation of a new gameserver. The user will be asked to enter all required options and parameters for that gameserver.

\section{Edit}
[include image]
The editing of a gameserver. The user will be able to edit the gameservers settings but not the game itself.

\section{List}
[include image]
The user sees a list of all gameservers that are under his control and sees some short status information of each server.

\section{Delete}
[include image]
The user may be able to delete the gameserver. This action cannot be undone, which is why he has to confirm this operation.

\chapter{Special Requirements}

\section{Owning an account}
The user has to be a registered user for our system.

\chapter{Preconditions}
\section{Must be logged in}
The user must be logged in to create, edit, see or delete a server.

\section{Being allowed to edit the server}
If the user wants to edit, see or delete a server, he must be a valid administrator for that specific server.

\chapter{Postconditions}

\section{Create}
After creating the gameserver the user will be redirected to the list overview, where the new entry will already be displayed.

\section{Edit}
After the user saved his edits, the updated data will be displayed in the servers detailed page.

\section{Delete}
After confirming the deletion modal, the gameserver will be permanently removed and no longer displayed in the list overview.

\end{document}